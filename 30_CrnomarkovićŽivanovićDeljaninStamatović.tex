

\documentclass[a4paper,12pt]{article}

\usepackage{color}
\usepackage{url}
\usepackage[T2A]{fontenc} 
\usepackage[utf8]{inputenc} 
\usepackage{graphicx}

\usepackage[english,serbian]{babel}

\usepackage[unicode]{hyperref}
\hypersetup{colorlinks,citecolor=green,filecolor=green,linkcolor=blue,urlcolor=blue}


\begin{document}

\title{\vspace{-8em}Pametni gradovi\\ \small{Seminarski rad u okviru kursa\\Tehničko i naučno pisanje\\ Matematički fakultet}}

\author{\hspace{0,9cm}Petar Deljanin \hspace{3,1cm}deljanin.petar2004@gmail.com\\ \hspace{0,2cm}Ana Crnomarković \hspace{3,2cm}anacrnomarkovic50@gmail.com\\ \hspace{0,3cm}Aleksandar Živanović \hspace{3cm}aleksandar.zivanovic39@gmail.com\\ \hspace{0,3cm}Nina Stamatović \hspace{3,2cm}nina.stamatovic03@gmail.com}
\date{11.~novembar 2022.}
\maketitle

\abstract{
Trenutno u gradovima, na planeti Zemlji, živi oko 4 milijarde ljudi, a sledećih 14 godina će se taj broj povećati za najmanje još jednu milijardu. 
Potrošnja električne energije, vodovod, odvoz smeća, zbrinjavanje otpadnih voda, zagađenje vazduha, manjak resursa, razni zdravstveni problemi, otežan saobraćaj – sve su to infrastrukturni problemi sa kojima se svet suočava i oni će još dodatno rasti u budućnosti. 
Rešenje su takozvani \textbf{pametni gradovi}, koji koriste postojeću i naprednu tehnologiju kako bi pokušali da promene način na koji živimo, putujemo, radimo i trošimo svoje slobodno vreme u gradovima. 
Gradovi i gradska infrastruktura su izuzetno kompleksni sistemi koji svakodnevno i učinkovito moraju odgovarati na mnogobrojne potrebe onih zbog kojih su nastali i koji ih čine živima, a to su građani. 
Sve to nameće potrebu za pronalaženjem inovativnih rešenja koja će rešiti sve probleme iznad navedene i upravo je to tema o kojoj se u poslednje vreme sve više govori i koja se nameće u svim segmentima urbanog života, i tom temom se mi bavimo u ovom radu.
Međutim, iako donosi razne prednosti, pametni gradovi nisu toliko savršeni i još uvek se radi na njihovom usavršavanju i realizovanju ideja i rešenja, ali sa osvrtom na bezbednost i sigurnost građana. Zato ćemo pored raznih prednosti navesti i neke mane koje prate koncept pametnih gradova. 

\tableofcontents

\newpage

\section{Uvod}
\label{sec:uvod}
Gradovi su glavna čvorišta ljudske i ekonomske aktivnosti. Oni imaju potencijal da stvore sinergiju\footnote{Sinergija je pojam koji opisuje stanje kada je celina nešto veće i drugačije od zbira svojih delova. } omogućavajući velike razvojne mogućnosti njihovim stanovnicima. Međutim, oni takođe stvaraju širok spektar problema koji mogu biti teško rešivi kako rastu u veličini i složenosti. Gradovi su i mesta gde su nejednakosti izraženije i, ako se njima ne upravlja, njihovi negativni efekti mogu prevazići pozitivne. \\

Urbana područja treba da upravljaju svojim razvojem, podržavajući ekonomsku konkurentnost, istovremeno jačajući socijalnu koheziju, održivost životne sredine i povećan kvalitet života svojih građana.\cite{referenca1} \\

Sa razvojem novih tehnoloških inovacija – uglavnom IKT-a\footnote{Informacione i komunikacione tehnologije} – koncept „pametnog grada“ se pojavljuje kao sredstvo za postizanje efikasnijih i održivijih gradova. \\

Od svog nastanka, koncept pametnog grada se razvio od izvođenja konkretnih projekata do sprovođenja globalnih strategija za rešavanje širih gradskih izazova. Stoga je neophodno dobiti sveobuhvatan pregled raspoloživih mogućnosti i povezati ih sa specifičnim izazovima grada.


\section{Koncept pametnih gradova}


Iako postoji neka vrsta konsenzusa\footnote{Konsenzus je jednolasno donošenje odluka.} da oznaka pametni grad predstavlja inovaciju u upravljanju gradom, njegovim uslugama i infrastrukturama, zajednička definicija pojma još nije data. Postoji širok spektar definicija šta bi pametni grad mogao biti. Međutim, dva trenda se mogu jasno razlikovati u vezi sa glavnim aspektima koje pametni gradovi moraju uzeti u obzir. \\

S jedne strane, postoji skup definicija koje stavljaju akcenat samo na jedan urbani aspekt, izostavljajući ostale okolnosti vezane za razvoj grada. Ova grupa monotopskih opisa pogrešno tumače da je krajnji cilj pametnog grada da obezbedi novi pristup urbanom upravljanju u kome se svi aspekti tretiraju uz međusobnu povezanost koja se dešava u praksi u gradu. Poboljšanje samo jednog dela urbanog ekosistema ne znači da se rešavaju problemi celine. \\

S druge strane, postoje neki autori koji naglašavaju kako je glavna razlika koncepta pametni grad povezanost svih urbanih aspekata. Zamršeni problemi između urbanizacije su istovremeno infrastrukturni, društveni i institucionalni i ovo preplitanje se ogleda u konceptu pametnog grada. Na primer, ako je grad pametan samo u pogledu njegove ekonomije, onda nije pametan u celini jer ne obraća pažnju na uslove života svojih građana, što i jeste cilj pametnih gradova. Iz definicija se može primetiti da je infrastruktura centralni deo pametnog grada i da je tehnologija ona koji to omogućava, ali je kombinacija, povezanost i integracija svih sistema ono što postaje osnovno da bi grad bio zaista pametan. Iz ovih definicija može se zaključiti da koncept pametnog grada podrazumeva sveobuhvatan pristup upravljanju i razvoju grada. Ove definicije opisuju ravnotežu tehnoloških, ekonomskih i društvenih faktora uključenih u urbani ekosistem.\cite{referenca2}\\



\section{Izazovi pametnih gradova}	
\label{sec:termini_i_citiranje}


 Kako gradovi i dalje neumorno rastu, njihove izazove treba pažljivo razmotriti kako bi se rast stanovništva, ekonomski razvoj i društveni napredak ujednačeno razvijali. Iako se većina globalnog BDP-a\footnote{Bruto domaći proizvod je ukupna vrednost proizvedenih krajnjih dobara i pruženih usluga u jednoj zemlji u određenom vremenskom periodu. } proizvodi u gradovima, sve što se dešava unutar njih ne podrazumeva pozitivne nuspojave. Gradovi su i mesta gde su nejednakosti izraženije i, ako se njima ne upravlja, negativni efekti mogu prevazići pozitivne. Model pametni grad može dovesti do boljeg planiranja i upravljanja gradom, a samim tim i do postizanja održivog modela urbanog razvoja. \\

U ASCIMER-ovoj\footnote{ASCIMER (Assessing Smart City Initiatives for the Mediterranean Region) je trogodišnji istraživački projekat koji je podržala Evropska investiciona banka u okviru EIBURS-a, a razvio ga je Politehnički univerzitet u Madridu.} prvoj godini rada, izazovi su identifikovani i klasifikovani u različite grupe kako bi se olakšali naredni koraci projekta. Analizirajući urbanu sredinu, istraživački radovi se bave različitim brojem oblasti za oblikovanje grada. Identifikovano je da se svi oni mogu rasporediti u okviru šest glavnih gradskih grupa: upravljanje, ekonomija, mobilnost, životna sredina, ljudi i život. \\


Oni predstavljaju specifične aspekte grada na koje pametne inicijative utiču da bi se postigli očekivani ciljevi pametnog grada. Ovo su neki ciljevi, tj. kako se život ljudi može poboljšati razvićem pametnih gradova:
\begin{enumerate}
\item Ukoliko žurite u grad i vozite se kolima – ne morate kretati ranije zbog traženja slobodnih parking mesta, jer aplikacija, zahvaljujući sistemu senzora za parking mesta, sama navodi na ona mesta koja su slobodna u realnom vremenu;
\item Štaviše, retko će se desiti situacija da se zaglavite u gužvi – semafori se prilagođavaju u zavisnosti od situacije u saobraćaju, ali i ukoliko je u nekoj ulici došlo do zastoja, aplikacija će  upozoriti da upravo nju izbegavate;
\item Redovi vožnje javnog prevoza se prilagođavaju u odnosu na mesto i vreme najvećih koncentracija putnika – tako da ne morate da brinete da li je gužva u autobusu ili ga neće biti kada stignete na stanicu. Svi podaci će biti dostupni na aplikaciji;
\item Kontejneri nikada nisu prepuni i gradom se ne širi nečistoća – zahvaljujući senzorima, gradske službe čistoće uvek imaju uvid u količinu otpada koji se nalazi u svakom kontejneru;
\item Ne morate brinuti o zagađenosti vazduha – grad ima ugrađen sistem senzora za merenje nivoa štetnih materija u vazduhu, te ukoliko je potrebno, šalju signale za mere koje je potrebno preduzeti.
\end{enumerate}
 \\



\begin{table}[h!]
\setlength{\arrayrulewidth}{0.5mm}
\setlength{\tabcolsep}{18pt}
\renewcommand{\arraystretch}{2.5}
    \large
    \centering
    \resizebox{1\textwidth}{!}{
    \begin{tabular}{|p{3cm}|p{3cm}|p{3cm}|p{3cm}|p{3cm}|p{3cm}|}
    \hline
   \textbf{Vladavina}    & \textbf{Ekonomija} & \textbf{Mobilnost} & \textbf{Okruženje} & \textbf{Ljudi} & \textbf{Život}\\
    \hline
     Fleksibilna vladavina & Nezaposlenost & Održiva mobilnost & Čuvanje energije & Nezaposlenost & Priuštiva stanarina\\
    \hline
    Gradovi koji se smanjuju & Gradovi koji se smanjuju & Inkluzivni gradovi & Gradovi koji se smanjuju & Solidarnost društva & Solidarnost društva\\
    \hline
    Ujedinjena teritorija & Ekonomski pad & Višemodalni transportni sistem & Duhovni pristup na sredinske i energetske probleme & Siromaštvo & Problemi sa srcem\\
    \hline 
    Kombinacija formalne i neformalne vlade & Ujedinjena teritorija & Urbani ekosistemi pod pritiskom & Urbani ekosistemi pod pritiskom & Starenje stanovništva & Hitno upravljanje\\
    \hline 
      & Jednosektorska ekonomija & Prometna gužva & Efekti promena klime & Druš. raznovrsnost kao izvor inovacija & Urbanizacija\\
     \hline
      & Održiva lokalna ekonommija & Mobilnost bez auta & Urbanizacija & Bezbednost na internetu & Bezbednost i sigurnost\\
      \hline 
       & Druš. raznovrsnost kao izvor inovacija & Deficit u ITC infrastrukturi & & & Bezbednost na internetu\\
       \hline 
       & Deficit u ITC infrastrukturi & & & & &
       \hline
    \end{tabular}
    }
    \caption{Izazovi u evropskim gradovima}
    \label{tab:my_label}
\end{table}


\newpage
Primeri pametnih gradova:
\begin{itemize}
\item \textbf{Evropa}: Barselona, London, Beč, Rejkjavik
\item \textbf{Severna Amerika}: Njujork, Kanzas, San Dijego, Kolambus
\item \textbf{Azija}: Hong Kong, Dubai, Tokio


 Kao primer, uzećemo London i Kolambus.
 

 \textbf{London}
 
 Za London možemo reći da je tehnološka prestonica Evrope. London beleži značajan rast populacije, što vrši pritisak na transport, zaštitu životne sredine, zdravstvo i upravljanje zagađenjem. Kako bi se rešio ovaj problem, London je pokrenuo niz inicijativa pod nazivom „Smarter London Together“, sa ciljem da London postane jedan od najpametnijih gradova u svetu. Projekat je usmeren na korisnike, deljenje podataka, povezivanje, digitalno vođstvo, veštine i saradnju između javnih službi i privatnog sektora.
Londonska kancelarija za tehnologiju i inovacije je ta koja je odgovorna da izgradi zajedničku platformu i pokrene priliku za saradnju i opseg digitalnih inovacija.
Oblasti koje su prioritetno razmatrane su:
Budućnost gradskog saobraćaja: London je lider u pametnoj mobilnosti, sa ambicioznim ciljem da do 2041. godine 80 procenata u glavnom gradu obavlja peške, biciklom ili javnim prevozom, da
saobraćaj postane ekološki prihvatljiv.
 Bez štetnih emisija: vlasti u Londonu su se obavezale da obezbede da London bude grad bez ugljenika do 2050 godine.
Vrednost sektora roba i usluga sa niskim prosekom ugljenika u 2017/18 iznosila je 39,7 milijardi funti, nakon što je u poslednje dve godine porasla za više od devet posto i očekuje se da će porasti za sedam posto do 2021/22 godne. 
Pristupačne javne usluge orijentisane prema građanima; Transformisanje javnih usluga da bi bolje zadovoljili potrebe svojih građana, a London je odavno prepoznat kao lider u mobilnoti, otvorenih podataka u javnu korist.

 \textbf{Kolambus }
 
 U leto 2017. godine, grad Кolambus u Ohaju započeo je inicijativu pametnog grada. Grad je sarađivao sa američkom elektranom za stvaranje grupe novih stanica za punjenje električnih vozila. Mnogi "pametni" gradovi, poput Kolambusa, koriste inicijative poput ove kako bi se pripremili za klimatske promene, proširili električnu infrastrukturu, zamene upotrebu javnog prevoza  električnim automobilima i time podstakli ljude da zajedno putuju na posao. Da bi to učinili, američko Ministarstvo saobraćaja dodelilo je Kolambusu zajam od 40 miliona dolara. Grad je takođe dobio 10 miliona dolara od njih.

Jedan od ključnih razloga zašto je komunalno preduzeće bilo uključeno u odabir lokacija za nove stanice za punjenje električnih vozila bilo je prikupljanje podataka.
Budući da su samovozeća vozila trenutno izložena "sve većem industrijskom istraživanju i zakonskim pritiscima širom sveta", stvaranje ruta i veza za njih je još jedan važan deo inicijative za njih.
\end{itemize} 



\section{Analiza projekata pametnih gradova}
\label{slike_i_tabele}

Različiti projekti pametnih gradova analizirani su na osnovu rezultata prethodne studije o konceptu pametnog grada i izazovima sa kojima se gradovi moraju suočiti. Analiza je podeljena u dve faze. Prvo je razvijen konceptualni okvir koji će se koristiti kao orijentacija kroz mogućnosti razvoja projekta pametnog grada u različitim već objašnjenim grupama. Drugo, napravljen je detaljan opis odabrane grupe projekata i gradova koji precizira: kojoj vrsti akcije projekta pametni grad pripada i koje su povezane gradske grupe koje on obuhvata; kakve gradske izazove pokušavaju da reše; i osnovne informacije o gradu u kojem je projekat sproveden. Osim toga, izrađeno je kratko objašnjenje samog projekta uključujući, kada je to moguće, stopu razvoja i obim projekta, kako se finansira, njegove ključne karakteristike, inovacije i njegove glavne uticaje. \\

Evolucija koncepta pametni grad vodi od pojedinačnih projekata, kao što je projekat održivog grada u Dubaiju, do globalnih gradskih strategija kroz koje je moguće odgovoriti na izazove grada na različitim nivoima (nacionalnom, regionalnom, međunarodnom). Stoga je uočeno da je neophodno razviti strategiju u okviru grada za kontrolisanje projekata u različitim grupama kako bi se postigla holistička i sveobuhvatna vizija.  Bez globalne strategije, grad je u opasnosti da izvede neke projekte koji dovode do toga da postane neuravnotežen, a samim tim i do drastičnog smanjenja uticaja ovih projekata. \\
Smanjenje uticaja dovodi i do negativnih posledica: 
\begin {enumerate}
\item Potrebna su značajna ulaganja u tehnologiju;
\item Postoji zavisnost od tehnoloških uslužnih kompanija;
\item Nekretnine postaju skuplje, jer je teže graditi i implementirati;
\item Između pametnih gradova i drugih gradova stvaraju se veliki tehnološki jazovi;
\item Značajno povećanje elektronskog otpada.
\end {enumerate}

\begin{figure}[h!]
    \centering
    \includegraphics[width=0.8\textwidth]{pametni grad}
    \caption{Komponente pametnog grada}\cite{referenca5}
    \label{fig:slika}
\end{figure}






\section{Zaključak}
\label{sec:zakljucak}

Kao složen koncept, nekoliko tipova projekata je definisano i zato postaje neophodno odabrati glavne karakteristike koje projekat pametni grad mora sadržati. Projekti pametnih gradova moraju biti višedimenzionalni i objediniti različita polja delovanja grada, u interakciji sa ljudskim i društvenim kapitalom. Pametni gradovi se moraju nositi sa izazovima koji se nameću u postizanju postavljenih ciljeva, u čemu im pomažu aktuelna tehnološka rešenja. Glavni ciljevi projekta pametni gradovi moraju biti rešavanje urbanih problema na efikasan način kako bi se postigli postavljeni ciljevi kao što su održivost grada, unapređenje transporta, smanjenje emisije štetnih gasova i uopšteno poboljšanje kvaliteta života njegovih stanovnika.\cite{referenca6}  \\

Projekti pametnih gradova moraju se pozabaviti problemima današnjih gradova, a istovremeno se osvrnuti na potencijalne probleme sa kojima će se gradovi suočiti u narednim decenijama. \\


\addcontentsline{toc}{section}{Literatura}
\appendix

\iffalse
\bibliography{seminarski} 
\bibliographystyle{plain}
\fi
\begin{thebibliography}{9}
\bibitem{referenca1} \emph{M. Batty et al., Smart Cities of the future. UCL Working Paper Series, 3rd Central European Conference in Regional Science} \\
\url{https://link.springer.com/article/10.1140/epjst/e2012-01703-3} 17.12.2022.
\bibitem{referenca2} \emph{L.M. Correia, Smart Cities Applications and Requirements, White Paper. Net!Works European Technology Platform, 2011.} \\
\url{https://grow.tecnico.ulisboa.pt/wp-content/uploads/2014/0/White_Paper_Smart_Cities_Applications.pdf} 17.12.2022.
\bibitem{referenca3} \emph{Directorate General for Regional Policy, European Commission.} \\
\url{https://en.wikipedia.org/wiki/Directorate-General_for_Regional_and_Urban_Policy} 17.12.2022.
\bibitem{referenca4} \emph{R. Giffinger et al., Smart Cities: Ranking of European Medium-Sized Cities, Vienna, Austria:Centre of Regional Science (SRF), Vienna University of Technology, 2007.} \\
\url{http://www.smart-cities.eu/download/smart_cities_final_report.pdf} 17.12.2022.
\bibitem{referenca5} \url{https://www.techtarget.com/iotagenda/definition/smart-city} 17.12.2022.
\bibitem{referenca6} \emph{C. Harrison et al., "Foundations for Smarter Cities", IBM Journal of Research and Development, vol. 54, no. 4, 2010.}\\
\url{https://ieeexplore.ieee.org/document/5512826} 17.12.2022.
\end{thebibliography}

\appendix
\end{document}
