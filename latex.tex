
\documentclass[a4paper,12pt]{article}

\usepackage{color}
\usepackage{url}
\usepackage[T2A]{fontenc} 
\usepackage[utf8]{inputenc} 
\usepackage{graphicx}

\usepackage[english,serbian]{babel}

\usepackage[unicode]{hyperref}
\hypersetup{colorlinks,citecolor=green,filecolor=green,linkcolor=blue,urlcolor=blue}

\newtheorem{primer}{Primer}[section]

\begin{document}

\title{Pametni gradovi\\ \small{Seminarski rad u okviru kursa\\Tehničko i naučno pisanje\\ Matematički fakultet}}

\author{Petar Deljanin\\ deljanin.petar2004@gmail.com\\Ana Crnomarković\\ anacrnomarkovic50@gmail.com\\ Aleksandar Živanović\\ aleksandar.zivanovic39@gmail.com\\ Nina Stamatović\\ nina.stamatovic03@gmail.com}
\date{11.~novembar 2022.}
\maketitle

\abstract{
U ovom tekstu je ukratko prikazana osnovna forma seminarskog rada. Obratite pažnju da je pored ove .pdf datoteke, u prilogu i odgovarajuća .tex datoteka, kao i .bib datoteka korišćena za generisanje literature. Na prvoj strani seminarskog rada su naslov, apstrakt i sadržaj, i to sve mora da stane na prvu stranu! Kako bi Vaš seminarski zadovoljio standarde i očekivanja, koristite uputstva i materijale sa predavanja na temu pisanja seminarskih radova. Ovo je samo šablon koji se odnosi na fizički izgled seminarskog rada (šablon koji \emph{morate} da ispoštujete!) kao i par tehničkih pomoćnih uputstava. 

\tableofcontents

\newpage

\section{Uvod}
\label{sec:uvod}
Gradovi su glavni polovi ljudske i ekonomske aktivnosti. Oni imaju potencijal da stvore sinergiju \footnote{Sinergija je pojam koji opisuje stanje kada je celina nešto veće i drugačije od zbira svojih delova. } omogućavajući velike razvojne mogućnosti njihovim stanovnicima. Međutim, oni takođe stvaraju širok spektar problema koji mogu biti teško rešiti kako rastu u veličini i složenosti. Gradovi su i mesta gde su nejednakosti jače i, ako se njima ne upravlja, njihovi negativni efekti mogu prevazići pozitivne. \\

Urbana područja treba da upravljaju svojim razvojem, podržavajući ekonomsku konkurentnost, istovremeno jačajući socijalnu koheziju, održivost životne sredine i povećan kvalitet života svojih građana. \\

Sa razvojem novih tehnoloških inovacija – uglavnom IKT-a \footnote{Informacione i komunikacione tehnologije}– koncept „pametnog grada“ se pojavljuje kao sredstvo za postizanje efikasnijih i održivijih gradova. \\

Od svog nastanka, koncept pametnog grada se razvio od izvođenja konkretnih projekata do sprovođenja globalnih strategija za rešavanje širih gradskih izazova. Stoga je neophodno dobiti sveobuhvatan pregled raspoloživih mogućnosti i povezati ih sa specifičnim izazovima grada.


\section{Koncept pametnih gradova}


Iako postoji neka vrsta konsenzusa \footnote{Konsenzus je jednolasno donošenje odluka.}da oznaka Pametni grad predstavlja inovaciju u upravljanju gradom, njegovim uslugama i infrastrukturama, zajednička definicija pojma još nije data. Postoji širok spektar definicija šta bi pametni grad mogao biti. Međutim, dva trenda se mogu jasno razlikovati u vezi sa glavnim aspektima koje pametni gradovi moraju uzeti u obzir. \\

S jedne strane, postoji skup definicija koje stavljaju akcenat samo na jedan urbani aspekt (tehnološki, ekološki, itd.), izostavljajući ostale okolnosti vezane za razvoj grada. Ova grupa monotopskih opisa pogrešno tumače da je krajnji cilj pametnog grada da obezbedi novi pristup urbanom upravljanju u kome se svi aspekti tretiraju uz međusobnu povezanost koja se dešava u praksi u gradu. Poboljšanje samo jednog dela urbanog ekosistema ne znači da se rešavaju problemi celine. \\

S druge strane, postoje neki autori koji naglašavaju kako je glavna razlika koncepta Pametni grad povezanost svih urbanih aspekata. Zamršeni problemi između urbanizacije su istovremeno infrastrukturni, društveni i institucionalni i ovo preplitanje se ogleda u konceptu pametnog grada. Iz definicija se može primetiti da je infrastruktura centralni deo Pametnog grada i da je tehnologija ona koji to omogućava, ali je kombinacija, povezanost i integracija svih sistema ono što postaje osnovno da bi grad bio zaista pametan. Iz ovih definicija može se zaključiti da koncept pametnog grada podrazumeva sveobuhvatan pristup upravljanju i razvoju grada. Ove definicije pokazuju ravnotežu tehnoloških, ekonomskih i društvenih faktora uključenih u urbani ekosistem. Definicije odražavaju holistički pristup \footnote{Holizam je univerzalno shvatanje da organizam u fiziološkom, psihološkom i socijalnom smislu može da funkcioniše samo kao celina. 
} na urbane probleme koristeći prednosti novih tehnologija tako da se urbani modeli i odnosi mogu redefinisati.\\
=======
Iako postoji neka vrsta konsenzusa da oznaka Pametni grad predstavlja inovaciju u upravljanju gradom, njegovim uslugama i infrastrukturama, zajednička definicija pojma još nije data. Postoji širok spektar definicija šta bi pametni grad mogao biti. Međutim, dva trenda se mogu jasno razlikovati u vezi sa glavnim aspektima koje pametni gradovi moraju uzeti u obzir. \\

S jedne strane, postoji skup definicija koje stavljaju akcenat samo na jedan urbani aspekt (tehnološki, ekološki, itd.), izostavljajući ostale okolnosti vezane za razvoj grada. Ova grupa monotopskih opisa pogrešno tumače da je krajnji cilj pametnog grada da obezbedi novi pristup urbanom upravljanju u kome se svi aspekti tretiraju uz međusobnu povezanost koja se dešava u praksi u gradu. Poboljšanje samo jednog dela urbanog ekosistema ne znači da se rešavaju problemi celine. \\

S druge strane, postoje neki autori koji naglašavaju kako je glavna razlika koncepta Pametni grad povezanost svih urbanih aspekata. Zamršeni problemi između urbanizacije su istovremeno infrastrukturni, društveni i institucionalni i ovo preplitanje se ogleda u konceptu pametnog grada. Iz definicija se može primetiti da je infrastruktura centralni deo Pametnog grada i da je tehnologija ona koji to omogućava, ali je kombinacija, povezanost i integracija svih sistema ono što postaje osnovno da bi grad bio zaista pametan. Iz ovih definicija može se zaključiti da koncept pametnog grada podrazumeva sveobuhvatan pristup upravljanju i razvoju grada. Ove definicije pokazuju ravnotežu tehnoloških, ekonomskih i društvenih faktora uključenih u urbani ekosistem. Definicije odražavaju holistički pristup na urbane probleme koristeći prednosti novih tehnologija tako da se urbani modeli i odnosi mogu redefinisati.\\




\section{Izazovi pametnih gradova}	
\label{sec:termini_i_citiranje}


 Kako gradovi i dalje neumorno rastu, njihove izazove treba pažljivo razmotriti kako bi se rast stanovništva, ekonomski razvoj i društveni napredak ujednačeno razvijali. Iako se većina globalnog BDP-a \footnote{Bruto domaći proizvod je ukupna vrednost proizvedenih krajnjih dobara i pruženih usluga u jednoj zemlji u određenom vremenskom periodu. } proizvodi u gradovima, sve što se dešava unutar ovih aglomeracija \footnote{Aglomeracija ili šire gradsko područje je prošireno područje grada koje se sastoji od centra i izvesnog broja prigradskih naselja koji zajedno sačinjavaju neprekinuto urbano područje.
} ne podrazumeva pozitivne nuspojave. Gradovi su i mesta gde su nejednakosti jače i, ako se njima ne upravlja, negativni efekti mogu prevazići pozitivne. Model Pametni grad može dovesti do boljeg planiranja i upravljanja gradom, a samim tim i do postizanja održivog modela urbanog razvoja. \\

U ASCIMER-ovoj \footnote{ASCIMER (Assessing Smart City Initiatives for the Mediterranean Region) je trogodišnji istraživački projekat koji je podržala Evropska investiciona banka u okviru EIBURS-a, a razvio ga je Politehnički univerzitet u Madridu.} prvoj godini rada, izazovi su identifikovani i klasifikovani u različite grupe kako bi se olakšali naredni koraci projekta. Analizirajući urbanu sredinu, istraživački radovi se bave različitim brojem oblasti za oblikovanje grada. U recenziranoj literaturi smo identifikovali da se svi oni mogu rasporediti u okviru šest glavnih gradskih grupa: upravljanje, ekonomija, mobilnost, životna sredina, ljudi i život. \\
=======
 Kako gradovi i dalje neumorno rastu, njihove izazove treba pažljivo razmotriti kako bi se rast stanovništva, ekonomski razvoj i društveni napredak ujednačeno razvijali. Iako se većina globalnog BDP-a proizvodi u gradovima, sve što se dešava unutar ovih aglomeracija ne podrazumeva pozitivne nuspojave. Gradovi su i mesta gde su nejednakosti jače i, ako se njima ne upravlja, negativni efekti mogu prevazići pozitivne. Model Pametni grad može dovesti do boljeg planiranja i upravljanja gradom, a samim tim i do postizanja održivog modela urbanog razvoja. \\

U ASCIMER-ovoj prvoj godini rada, izazovi su identifikovani i klasifikovani u različite grupe kako bi se olakšali naredni koraci projekta. Analizirajući urbanu sredinu, istraživački radovi se bave različitim brojem oblasti za oblikovanje grada. U recenziranoj literaturi smo identifikovali da se svi oni mogu rasporediti u okviru šest glavnih gradskih grupa: upravljanje, ekonomija, mobilnost, životna sredina, ljudi i život. \\


Oni predstavljaju specifične aspekte grada na koje pametne inicijative utiču da bi se postigli očekivani ciljevi strategije pametnog grada (održivost, efikasnost i visok kvalitet života). Tehnologija se sama po sebi ne smatra poljem delovanja, već sredstvom koje poboljšava efikasnost projekata. \\

Unutar svake od grupa identifikovani su različiti gradski izazovi kako za gradove severnog Mediterana, tako i za gradove južnog i istočnog Mediterana. Gradovi za koje ćemo ovom prilikom smatrati da pripadaju regionu severnog Mediterana su oni koji se nalaze u zemljama Evropske unije. Zemlje u regionu južnog i istočnog Mediterana koje su razmatrane u studiji su: Maroko, Alžir, Tunis, Libija, Egipat, Jordan, Izrael, Liban, Sirija i Turska. \\

Ukupno je identifikovano dvadeset devet izazova za gradove severnog regiona. Među njima, dvadeset se odnosi na samo jednu grupu. Devet izazova se odnosi na više grupa. Za južne gradove identifikovano je dvadeset izazova, od kojih se jedanaest odnosi na samo jednu grupu, dok ostalih devet odgovara dve ili više.\\

\begin{table}[h!]
    \centering
    \resizebox{1\textwidth}{!}{
    \begin{tabular}{|p{4cm}|p{4cm}|p{4cm}|p{4cm}|p{4cm}|p{4cm}|}
    \hline
   \textbf{Vladavina}    & \textbf{Ekonomija} & \textbf{Mobilnost} & \textbf{Okruženje} & \textbf{Ljudi} & \textbf{Život}\\
    \hline
     Fleksibilna vladavina & Nezaposlenost & Održiva mobilnost & Čuvanje energije & Nezaposlenost & Priuštiva stanarina\\
    \hline
    Gradovi koji se smanjuju & Gradovi koji se smanjuju & Inkluzivni gradovi & Gradovi koji se smanjuju & Solidarnost društva & Solidarnost društva\\
    \hline
    Ujedinjena teritorija & Ekonomski pad & Višemodalni transportni sistem & Duhovni pristup na sredinske i energetske probleme & Siromašnost & Problemi sa srcem\\
    \hline 
    Kombinacija formalne i neformalne vlade & Ujedinjena teritorija & Urbani ekosistemi pod pritiskom & Urbani ekosistemi pod pritiskom & Starenje stanovništva & Hitno upravljanje\\
    \hline 
      & Jednosektorska ekonomija & Prometna gužva & Efekti promena klime & Druš. raznovrsnost kao izvor inovacija & Urbanizacija\\
     \hline
      & Održiva lokalna ekonommija & Mobilnost bez auta & Urbanizacija & Bezbednost na internetu & Bezbednost i sigurnost\\
      \hline 
       & Druš. raznovrsnost kao izvor inovacija & Deficit u ITC infrastrukturi & & & Bezbednost na internetu\\
       \hline 
       & Deficit u ITC infrastrukturi & & & & &
       \hline
    \end{tabular}
    }
    \caption{Izazovi u evropskim gradovima.}
    \label{tab:my_label}
\end{table}






\section{Analiza projekata pametnih gradova}
\label{slike_i_tabele}

Različiti projekti pametnih gradova analizirani su na osnovu rezultata prethodne studije o konceptu pametnog grada i izazovima sa kojima se gradovi moraju suočiti. Analiza je podeljena u dve faze; prvo je razvijen konceptualni okvir koji će se koristiti kao orijentacija kroz mogućnosti razvoja projekta Pametni grad u različitim već objašnjenim grupama. Drugo, detaljan opis odabrane grupe projekata i gradova koji precizira: kojoj vrsti akcije projekta Pametni grad pripada i koje su povezane gradske grupe koje on obuhvata; kakve gradske izazove pokušavaju da reše; i osnovne informacije o gradu u kojem je projekat sproveden. Osim toga, izrađeno je kratko objašnjenje samog projekta uključujući, kada je to moguće, stopu razvoja i obim projekta; kako se finansira; njegove ključne karakteristike inovacije i njegove glavne uticaje. \\

Evolucija koncepta Pametni grad vodi od pojedinačnih projekata do globalnih gradskih strategija kroz koje je moguće odgovoriti na izazove grada na različitim nivoima (nacionalnom, regionalnom, međunarodnom). Stoga je uočeno da je neophodno razviti strategiju u okviru grada za kontrolisanje projekata u različitim grupama kako bi se postigla holistička i sveobuhvatna vizija. Shodno tome, pored analize izdvojenih radnji, identifikovane su i analizirane i neke izvanredne mediteranske strategije. Podela grada u 6 grupa je presudna za dobar učinak. Bez globalne strategije, grad je u opasnosti da izvede neke projekte koji dovode do toga da postane neuravnotežen, a samim tim i do drastičnog smanjenja uticaja ovih projekata. \\

Sve ove informacije su prikupljene u Vodiču za projekte koji tek treba da bude objavljen. Međutim, u ovom radu biće prikazani samo rezultati konceptualnog okvira i analiza nezavisnih projekata.\\



\begin{table}[h!]
\begin{center}
\caption{Razlčita poravnanja u okviru iste tabele ne treba koristiti jer su nepregledna.}
\begin{tabular}{|c|l|r|} \hline
centralno poravnanje& levo poravnanje& desno poravnanje\\ \hline
a &b&c\\ \hline
d &e&f\\ \hline
\end{tabular}
\label{tab:tabela1}
\end{center}
\end{table}

\\



\section{Zaključak}
\label{sec:zakljucak}
Tokom ove prve godine, ASCIMER projekat je bio fokusiran na razvoj konceptualnog okvira za metodologiju procene za projekte Pametnih gradova u regionu Mediterana. Razumevanje i klasifikacija polja delovanja i postojećih projekata Pametnog grada bili su glavni rezultati projekta ove godine. \\

Kao složen koncept, nekoliko tipova projekata je definisano i zato postaje neophodno odabrati glavne karakteristike koje projekat Pametni grad mora sadržati. Projekti pametnih gradova moraju biti višedimenzionalni i objediniti različita polja delovanja grada, u interakciji sa ljudskim i društvenim kapitalom. Tehnološka rešenja se moraju shvatiti kao sredstvo za postizanje ciljeva pametnih gradova i za nošenje sa izazovima sa kojima se gradovi moraju suočiti. Glavni ciljevi projekta Pametni gradovi moraju biti rešavanje urbanih problema na efikasan način kako bi se poboljšala održivost grada i kvalitet života njegovih stanovnika.  \\

Glavni zahtev za projekte pametnih gradova mora biti rešavanje stvarnih izazova sa kojima će se gradovi suočiti u budućnosti. Ovo je prvi korak koji metodologija procene mora uzeti u obzir. Kada se analizira mediteranski region, ključno je razumeti različite izazove sa kojima se gradovi u južnom i istočno-mediteranskom regionu moraju suočiti i na koji način su oni povezani. Projekti pametnih gradova moraju se pozabaviti problemima današnjih gradova, a istovremeno se osvrnuti na potencijalne probleme sa kojima će se gradovi suočiti u narednim decenijama. \\

Metodologije procene su neophodne za razmatranje stvarnog uticaja projekata. Klasifikacija postojećih rešenja i projekata je glavni korak za postavljanje aspekata koje metodologija mora da uzme u obzir. Ovi aspekti moraju biti vezani za prethodno definisane izazove, razumejući na koji način daju rešenje za probleme grada. Pružanje primera u svakoj od oblasti, u vezi sa ovim projektnim aktivnostima i izazovima, rezultira alatom za razvoj rešenja za probleme grada sa višedimenzionalnim i sveobuhvatnim pristupom. \\

Uzimajući u obzir podatke o izazovima grada i analizu projekata ove prve godine, biće razvijena metodologija procene kako bi se razmotrili projekti pametnog grada u regionu južnog i istočnog Mediterana. Razvoj pokazitelja prilagođenih glavnim karakteristikama, projektima i izazovima gradova mediteranskog regiona i definisanje odnosa između njih za razvoj ispravne metodologije biće glavni cilj ASCIMER-a tokom naredne godine.\\
