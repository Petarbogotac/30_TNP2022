\documentclass{beamer}
\usepackage{beamerthemeshadow}
\usepackage{graphicx}
\usepackage{color}
\usepackage[utf8]{inputenc}
\usepackage{hyperref}
\usepackage{caption}
\usepackage[flushleft]{threeparttable}
\usepackage[english, serbian]{babel}
\usepackage{subfigure}
\definecolor{caribbeangreen}{rgb}{0.6, 0.33, 0.73}
\setbeamercolor{structure}{fg=caribbeangreen}
\captionsetup[figure]{labelformat=empty}
\captionsetup[table]{labelformat=empty}


\def\d{{\fontencoding{T1}\selectfont\dj}}
\def\D{{\fontencoding{T1}\selectfont\DJ}}


\title{Tehničko i naučno pisanje}
\subtitle{-- Pametni gradovi --}
\author{Ana Crnomarković \and Petar Deljanin\\ \and Nina Stamatović \and Aleksandar Živanović}
\institute{Matematički fakultet\\Univerzitet u Beogradu}
\date{
	\footnotesize{Beograd, 2022.}	
}

\begin{document}
\begin{frame}
	\thispagestyle{empty}
	\titlepage
\end{frame}

\addtocounter{framenumber}{-1}

\begin{frame}[fragile]\frametitle{Literatura}
	\begin{itemize}
		\item Zasnovano na seminarskom radu "Pametni gradovi - Petar Deljanin, Ana Crnomarković, Aleksandar Živanović, Nina Stamatović", koji se može naći na sledećem linku:
		(\url{https://raw.githubusercontent.com/Petarbogotac/30_TNP2022/main/30_Crnomarkovi%C4%87%C5%BDivanovi%C4%87DeljaninStamatovi%C4%87.pdf.pdf})
	\end{itemize}
\end{frame}
\begin{frame}
	\frametitle{Pregled} % Table of contents slide, comment this block out to remove it
	\tableofcontents[] 
\end{frame}
\section{Uvod}

\begin{frame}[fragile]\frametitle{Uvod}
	\begin{itemize}	
		  \item{Glavna područja}
           \item{Problemi}
           \item{IKT}
           \item{Od jednostavnih projekata do globalnih strategija}
	\end{itemize}
 \begin{figure}[h!]
        \centering\includegraphics[height=3cm]{abt-img.jpg} 
\end{figure}
 
\end{frame}

\section{Koncept pametnih gradova}


\begin{frame}[fragile]\frametitle{Koncept pametih gradova}
	\begin{itemize}	
		 \item{Šta su pametni gradovi?} 
          \item{Postoje dva modela}
         \item{Međusobna povezanost nije dovoljna}

	\end{itemize}
   \begin{figure}[h!]
        \centering\includegraphics[height=4cm]{concept.jpg} 
        \caption{\emph{Komponente}}
        \label{fig:Komponente}
\end{figure}
\end{frame}

\section{Izazovi pametnih gradova}
\begin{frame}[fragile]\frametitle{Izazovi pametnih gradova}
\begin{itemize}
    \item Izazovi su identifikovani i klasifikovani u različite grupe
    \item Oblasti oblikovanja gradova:   \\
    - upravljanje, mobilnost, ekonomija, životna sredina, ljudi i život
    \begin{figure}[h!]
        \centering\includegraphics[height=3cm]{Picture1.png} 
\end{figure}
    \item Različiti ciljevi
\end{itemize}

\end{frame}



\section{Analiza projekata pametnih gradova}
\begin{frame}[fragile]\frametitle{Analiza projekata pametnih gradova}

\begin{itemize}
    \item Podeljena je u dve faze\\
    1. Razvoj konceptualnog okvira\\
    2. Detaljan opis
    \item Razvoj strategije
    \item Negativne posledice
    \begin{figure}[h!]
        \centering
        \includegraphics[height=3cm]{pametni gradovi.jpg} 
\end{figure}
\end{itemize}

\end{frame}
\begin{frame}{Primeri pametnih gradova}
\begin{table}[h!]
\begin{center}
\begin{tabular}{|c|l|} \hline
\textbf{Kontinent}& \textbf{Gradovi}\\ \hline
Evropa &Barselona, London, Beč, Rejkjavik\\ \hline
Severna Amerika &Njujork, Kanzas, San Dijego, Kolambus\\ \hline
Azija &Hong Kong, Dubai, Tokio\\ \hline

\end{tabular}
\label{tab:tabela1}
\end{center}
\end{table}
\end{frame}

\section{Zaključak}

\begin{frame}[fragile]\frametitle{Zaključak}
	\begin{itemize}
	\item Projekti pametnih gradova moraju biti višedimenzionalni i objediniti različita polja delovanja grada, u interakciji sa ljudskim i drušvenim    kapitalom. 
    	\item Takođe se moraju pozabaviti problemima današnjih gradova, a istovremeno se osvrnuti na potencijalne probleme sa kojima  ́će se gradovi suočiti u narednim decenijama.
	\end{itemize}
\end{frame}

\end{document}
